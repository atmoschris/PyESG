% Generated by Sphinx.
\def\sphinxdocclass{report}
\documentclass[letterpaper,10pt,english]{sphinxmanual}
\usepackage[utf8]{inputenc}
\DeclareUnicodeCharacter{00A0}{\nobreakspace}
\usepackage{cmap}
\usepackage[T1]{fontenc}
\usepackage{babel}
\usepackage{times}
\usepackage[Bjarne]{fncychap}
\usepackage{longtable}
\usepackage{sphinx}
\usepackage{multirow}

\addto\captionsenglish{\renewcommand{\figurename}{Fig. }}
\addto\captionsenglish{\renewcommand{\tablename}{Table }}
\floatname{literal-block}{Listing }



\title{PyESG Documentation}
\date{July 10, 2015}
\release{0.0.13}
\author{Feng Zhu}
\newcommand{\sphinxlogo}{}
\renewcommand{\releasename}{Release}
\makeindex

\makeatletter
\def\PYG@reset{\let\PYG@it=\relax \let\PYG@bf=\relax%
    \let\PYG@ul=\relax \let\PYG@tc=\relax%
    \let\PYG@bc=\relax \let\PYG@ff=\relax}
\def\PYG@tok#1{\csname PYG@tok@#1\endcsname}
\def\PYG@toks#1+{\ifx\relax#1\empty\else%
    \PYG@tok{#1}\expandafter\PYG@toks\fi}
\def\PYG@do#1{\PYG@bc{\PYG@tc{\PYG@ul{%
    \PYG@it{\PYG@bf{\PYG@ff{#1}}}}}}}
\def\PYG#1#2{\PYG@reset\PYG@toks#1+\relax+\PYG@do{#2}}

\expandafter\def\csname PYG@tok@ge\endcsname{\let\PYG@it=\textit}
\expandafter\def\csname PYG@tok@k\endcsname{\let\PYG@bf=\textbf\def\PYG@tc##1{\textcolor[rgb]{0.00,0.44,0.13}{##1}}}
\expandafter\def\csname PYG@tok@kn\endcsname{\let\PYG@bf=\textbf\def\PYG@tc##1{\textcolor[rgb]{0.00,0.44,0.13}{##1}}}
\expandafter\def\csname PYG@tok@cp\endcsname{\def\PYG@tc##1{\textcolor[rgb]{0.00,0.44,0.13}{##1}}}
\expandafter\def\csname PYG@tok@na\endcsname{\def\PYG@tc##1{\textcolor[rgb]{0.25,0.44,0.63}{##1}}}
\expandafter\def\csname PYG@tok@gr\endcsname{\def\PYG@tc##1{\textcolor[rgb]{1.00,0.00,0.00}{##1}}}
\expandafter\def\csname PYG@tok@no\endcsname{\def\PYG@tc##1{\textcolor[rgb]{0.38,0.68,0.84}{##1}}}
\expandafter\def\csname PYG@tok@gh\endcsname{\let\PYG@bf=\textbf\def\PYG@tc##1{\textcolor[rgb]{0.00,0.00,0.50}{##1}}}
\expandafter\def\csname PYG@tok@sr\endcsname{\def\PYG@tc##1{\textcolor[rgb]{0.14,0.33,0.53}{##1}}}
\expandafter\def\csname PYG@tok@nc\endcsname{\let\PYG@bf=\textbf\def\PYG@tc##1{\textcolor[rgb]{0.05,0.52,0.71}{##1}}}
\expandafter\def\csname PYG@tok@mo\endcsname{\def\PYG@tc##1{\textcolor[rgb]{0.13,0.50,0.31}{##1}}}
\expandafter\def\csname PYG@tok@se\endcsname{\let\PYG@bf=\textbf\def\PYG@tc##1{\textcolor[rgb]{0.25,0.44,0.63}{##1}}}
\expandafter\def\csname PYG@tok@c\endcsname{\let\PYG@it=\textit\def\PYG@tc##1{\textcolor[rgb]{0.25,0.50,0.56}{##1}}}
\expandafter\def\csname PYG@tok@nl\endcsname{\let\PYG@bf=\textbf\def\PYG@tc##1{\textcolor[rgb]{0.00,0.13,0.44}{##1}}}
\expandafter\def\csname PYG@tok@nv\endcsname{\def\PYG@tc##1{\textcolor[rgb]{0.73,0.38,0.84}{##1}}}
\expandafter\def\csname PYG@tok@ss\endcsname{\def\PYG@tc##1{\textcolor[rgb]{0.32,0.47,0.09}{##1}}}
\expandafter\def\csname PYG@tok@mb\endcsname{\def\PYG@tc##1{\textcolor[rgb]{0.13,0.50,0.31}{##1}}}
\expandafter\def\csname PYG@tok@w\endcsname{\def\PYG@tc##1{\textcolor[rgb]{0.73,0.73,0.73}{##1}}}
\expandafter\def\csname PYG@tok@il\endcsname{\def\PYG@tc##1{\textcolor[rgb]{0.13,0.50,0.31}{##1}}}
\expandafter\def\csname PYG@tok@gs\endcsname{\let\PYG@bf=\textbf}
\expandafter\def\csname PYG@tok@kp\endcsname{\def\PYG@tc##1{\textcolor[rgb]{0.00,0.44,0.13}{##1}}}
\expandafter\def\csname PYG@tok@gd\endcsname{\def\PYG@tc##1{\textcolor[rgb]{0.63,0.00,0.00}{##1}}}
\expandafter\def\csname PYG@tok@gp\endcsname{\let\PYG@bf=\textbf\def\PYG@tc##1{\textcolor[rgb]{0.78,0.36,0.04}{##1}}}
\expandafter\def\csname PYG@tok@gi\endcsname{\def\PYG@tc##1{\textcolor[rgb]{0.00,0.63,0.00}{##1}}}
\expandafter\def\csname PYG@tok@nn\endcsname{\let\PYG@bf=\textbf\def\PYG@tc##1{\textcolor[rgb]{0.05,0.52,0.71}{##1}}}
\expandafter\def\csname PYG@tok@kt\endcsname{\def\PYG@tc##1{\textcolor[rgb]{0.56,0.13,0.00}{##1}}}
\expandafter\def\csname PYG@tok@m\endcsname{\def\PYG@tc##1{\textcolor[rgb]{0.13,0.50,0.31}{##1}}}
\expandafter\def\csname PYG@tok@s2\endcsname{\def\PYG@tc##1{\textcolor[rgb]{0.25,0.44,0.63}{##1}}}
\expandafter\def\csname PYG@tok@kd\endcsname{\let\PYG@bf=\textbf\def\PYG@tc##1{\textcolor[rgb]{0.00,0.44,0.13}{##1}}}
\expandafter\def\csname PYG@tok@nt\endcsname{\let\PYG@bf=\textbf\def\PYG@tc##1{\textcolor[rgb]{0.02,0.16,0.45}{##1}}}
\expandafter\def\csname PYG@tok@kr\endcsname{\let\PYG@bf=\textbf\def\PYG@tc##1{\textcolor[rgb]{0.00,0.44,0.13}{##1}}}
\expandafter\def\csname PYG@tok@mh\endcsname{\def\PYG@tc##1{\textcolor[rgb]{0.13,0.50,0.31}{##1}}}
\expandafter\def\csname PYG@tok@ni\endcsname{\let\PYG@bf=\textbf\def\PYG@tc##1{\textcolor[rgb]{0.84,0.33,0.22}{##1}}}
\expandafter\def\csname PYG@tok@gu\endcsname{\let\PYG@bf=\textbf\def\PYG@tc##1{\textcolor[rgb]{0.50,0.00,0.50}{##1}}}
\expandafter\def\csname PYG@tok@sc\endcsname{\def\PYG@tc##1{\textcolor[rgb]{0.25,0.44,0.63}{##1}}}
\expandafter\def\csname PYG@tok@sx\endcsname{\def\PYG@tc##1{\textcolor[rgb]{0.78,0.36,0.04}{##1}}}
\expandafter\def\csname PYG@tok@ow\endcsname{\let\PYG@bf=\textbf\def\PYG@tc##1{\textcolor[rgb]{0.00,0.44,0.13}{##1}}}
\expandafter\def\csname PYG@tok@si\endcsname{\let\PYG@it=\textit\def\PYG@tc##1{\textcolor[rgb]{0.44,0.63,0.82}{##1}}}
\expandafter\def\csname PYG@tok@vi\endcsname{\def\PYG@tc##1{\textcolor[rgb]{0.73,0.38,0.84}{##1}}}
\expandafter\def\csname PYG@tok@sd\endcsname{\let\PYG@it=\textit\def\PYG@tc##1{\textcolor[rgb]{0.25,0.44,0.63}{##1}}}
\expandafter\def\csname PYG@tok@nf\endcsname{\def\PYG@tc##1{\textcolor[rgb]{0.02,0.16,0.49}{##1}}}
\expandafter\def\csname PYG@tok@sh\endcsname{\def\PYG@tc##1{\textcolor[rgb]{0.25,0.44,0.63}{##1}}}
\expandafter\def\csname PYG@tok@nd\endcsname{\let\PYG@bf=\textbf\def\PYG@tc##1{\textcolor[rgb]{0.33,0.33,0.33}{##1}}}
\expandafter\def\csname PYG@tok@kc\endcsname{\let\PYG@bf=\textbf\def\PYG@tc##1{\textcolor[rgb]{0.00,0.44,0.13}{##1}}}
\expandafter\def\csname PYG@tok@mf\endcsname{\def\PYG@tc##1{\textcolor[rgb]{0.13,0.50,0.31}{##1}}}
\expandafter\def\csname PYG@tok@ne\endcsname{\def\PYG@tc##1{\textcolor[rgb]{0.00,0.44,0.13}{##1}}}
\expandafter\def\csname PYG@tok@vg\endcsname{\def\PYG@tc##1{\textcolor[rgb]{0.73,0.38,0.84}{##1}}}
\expandafter\def\csname PYG@tok@s\endcsname{\def\PYG@tc##1{\textcolor[rgb]{0.25,0.44,0.63}{##1}}}
\expandafter\def\csname PYG@tok@bp\endcsname{\def\PYG@tc##1{\textcolor[rgb]{0.00,0.44,0.13}{##1}}}
\expandafter\def\csname PYG@tok@c1\endcsname{\let\PYG@it=\textit\def\PYG@tc##1{\textcolor[rgb]{0.25,0.50,0.56}{##1}}}
\expandafter\def\csname PYG@tok@cm\endcsname{\let\PYG@it=\textit\def\PYG@tc##1{\textcolor[rgb]{0.25,0.50,0.56}{##1}}}
\expandafter\def\csname PYG@tok@vc\endcsname{\def\PYG@tc##1{\textcolor[rgb]{0.73,0.38,0.84}{##1}}}
\expandafter\def\csname PYG@tok@go\endcsname{\def\PYG@tc##1{\textcolor[rgb]{0.20,0.20,0.20}{##1}}}
\expandafter\def\csname PYG@tok@o\endcsname{\def\PYG@tc##1{\textcolor[rgb]{0.40,0.40,0.40}{##1}}}
\expandafter\def\csname PYG@tok@s1\endcsname{\def\PYG@tc##1{\textcolor[rgb]{0.25,0.44,0.63}{##1}}}
\expandafter\def\csname PYG@tok@cs\endcsname{\def\PYG@tc##1{\textcolor[rgb]{0.25,0.50,0.56}{##1}}\def\PYG@bc##1{\setlength{\fboxsep}{0pt}\colorbox[rgb]{1.00,0.94,0.94}{\strut ##1}}}
\expandafter\def\csname PYG@tok@err\endcsname{\def\PYG@bc##1{\setlength{\fboxsep}{0pt}\fcolorbox[rgb]{1.00,0.00,0.00}{1,1,1}{\strut ##1}}}
\expandafter\def\csname PYG@tok@gt\endcsname{\def\PYG@tc##1{\textcolor[rgb]{0.00,0.27,0.87}{##1}}}
\expandafter\def\csname PYG@tok@nb\endcsname{\def\PYG@tc##1{\textcolor[rgb]{0.00,0.44,0.13}{##1}}}
\expandafter\def\csname PYG@tok@mi\endcsname{\def\PYG@tc##1{\textcolor[rgb]{0.13,0.50,0.31}{##1}}}
\expandafter\def\csname PYG@tok@sb\endcsname{\def\PYG@tc##1{\textcolor[rgb]{0.25,0.44,0.63}{##1}}}

\def\PYGZbs{\char`\\}
\def\PYGZus{\char`\_}
\def\PYGZob{\char`\{}
\def\PYGZcb{\char`\}}
\def\PYGZca{\char`\^}
\def\PYGZam{\char`\&}
\def\PYGZlt{\char`\<}
\def\PYGZgt{\char`\>}
\def\PYGZsh{\char`\#}
\def\PYGZpc{\char`\%}
\def\PYGZdl{\char`\$}
\def\PYGZhy{\char`\-}
\def\PYGZsq{\char`\'}
\def\PYGZdq{\char`\"}
\def\PYGZti{\char`\~}
% for compatibility with earlier versions
\def\PYGZat{@}
\def\PYGZlb{[}
\def\PYGZrb{]}
\makeatother

\renewcommand\PYGZsq{\textquotesingle}

\begin{document}

\maketitle
\tableofcontents
\phantomsection\label{index::doc}


Contents:


\chapter{Introduction}
\label{introduction:pyesg-documentation}\label{introduction:introduction}\label{introduction::doc}
The Python Library of Earth Spherical Geometry (PyESG) is aiming to perform spherical geometry on the earth and interpolate data from an unstructured mesh grid to another one, which is the so-called ``regridding'' or ``remapping''.
Thus two main usages of PyESG are:
\begin{itemize}
\item {} 
Spherical geometry

\item {} 
Regridding

\end{itemize}

and the second usage (regridding) is the main goal of this library.

There are some existing implementations of regridding, such as \href{http://www.ncl.ucar.edu}{NCL} and \href{http://www.earthsystemmodeling.org}{EMSF}.
\href{http://www.ncl.ucar.edu}{NCL} mainly performs interpolation between regular mesh grids, and recently (after the version of \href{http://www.ncl.ucar.edu/Document/Functions/ESMF/ESMF\_regrid.shtml}{6.1.0}) it concludes the implementation of \href{http://www.earthsystemmodeling.org}{EMSF}, which performs interpolation between any mesh grids.
However, the APIs of \href{http://www.earthsystemmodeling.org}{EMSF} are ported to specific global models that it is not so convinient to be used in a lightweighted way.
By a lightweighted way I mean calling a function like:

\begin{Verbatim}[commandchars=\\\{\}]
\PYG{n}{func}\PYG{p}{(}\PYG{n}{data}\PYG{p}{,} \PYG{n}{mesh\PYGZus{}old}\PYG{p}{,} \PYG{n}{mesh\PYGZus{}new}\PYG{p}{)}
\end{Verbatim}

Therefore, PyESG is developed to make regridding between any mesh grids easier.


\chapter{Spherical Geometry}
\label{spherical_geometry:emsf}\label{spherical_geometry::doc}\label{spherical_geometry:spherical-geometry}
Spherical geometry is a subject about the geometry on a sphere.
In this library, basic spherical geometries are introduced to serve the goal of regridding, so it may be not comprehensive for now.
The main fuctionalities of spherical geometry in this library are introduced below.


\section{Distance between two points}
\label{spherical_geometry:distance-between-two-points}
This is the so-called {}`{}`


\section{Waypoints during an arc}
\label{spherical_geometry:waypoints-during-an-arc}

\section{Intersection betweeen two arcs}
\label{spherical_geometry:intersection-betweeen-two-arcs}

\section{Angle between two arcs}
\label{spherical_geometry:angle-between-two-arcs}

\section{Triangle area}
\label{spherical_geometry:triangle-area}

\section{Triangle inside test}
\label{spherical_geometry:triangle-inside-test}

\section{Quadrangle area}
\label{spherical_geometry:quadrangle-area}

\section{Quadrangle inside test}
\label{spherical_geometry:quadrangle-inside-test}

\chapter{Regridding}
\label{regridding::doc}\label{regridding:regridding}
This section will introduce the algorithm for regridding in the PyESG.


\section{Basic Idea}
\label{regridding:basic-idea}
The basic idea of regridding between two mesh grids, including the case of unstructured mesh grids, is interpolation.
Spherical interpolation is very complicated, and it is assumed in PyESG that for a small regional area, planar interpolation is appropriage with limit error.
Since the mesh grid is unstructured, the interpolation method used is \textbf{barycentric interpolation}.

To make the API easy to use, the Python function of regridding is like below:

\begin{Verbatim}[commandchars=\\\{\}]
\PYG{n}{pyesg}\PYG{o}{.}\PYG{n}{Interp}\PYG{o}{.}\PYG{n}{regrid}\PYG{p}{(}\PYG{n}{mesh\PYGZus{}old}\PYG{p}{,} \PYG{n}{mesh\PYGZus{}new}\PYG{p}{)}
\end{Verbatim}

This function dose not regrid a specific data, but calculate the regridding information when one performs regridding from \code{mesh\_old} to \code{mesh\_new}.
If the mesh grids are two-dimensional, say \code{(nlat, nlon)}, then the shape of the return is \code{(3, nlat, nlon)}, which stores the location information and weights of the three points that used to calculate the interpolated result.

So how to realize this function?
\begin{itemize}
\item {} 
Step 1: {\hyperref[regridding:search]{\emph{Search}}}

\end{itemize}

To calculate the interpolated value of a specific point \code{(lat, lon)} in the \code{mesh\_new}, we need to find out the three points in the \code{mesh\_old} used for interpolation.
\begin{itemize}
\item {} 
Step 2: {\hyperref[regridding:interpolation]{\emph{Interpolation}}}

\end{itemize}

After those three points are found, the barycentric interpolation can be performed.


\section{Search}
\label{regridding:search}

\section{Interpolation}
\label{regridding:interpolation}

\chapter{Souce code}
\label{code:souce-code}\label{code:module-pyesg}\label{code::doc}\index{pyesg (module)}\index{Arc (class in pyesg)}

\begin{fulllineitems}
\phantomsection\label{code:pyesg.Arc}\pysiglinewithargsret{\strong{class }\code{pyesg.}\bfcode{Arc}}{\emph{p1}, \emph{p2}}{}
An arc on the earth defined by two points p1 and p2.
It also can be seen as the relationship between two points.
\index{distance() (pyesg.Arc method)}

\begin{fulllineitems}
\phantomsection\label{code:pyesg.Arc.distance}\pysiglinewithargsret{\bfcode{distance}}{}{}
Calculate the great-circle distance of the arc.

{[}reference: \href{http://en.wikipedia.org/wiki/Great-circle\_distance}{http://en.wikipedia.org/wiki/Great-circle\_distance}{]}

\end{fulllineitems}

\index{rad() (pyesg.Arc method)}

\begin{fulllineitems}
\phantomsection\label{code:pyesg.Arc.rad}\pysiglinewithargsret{\bfcode{rad}}{}{}
Convert great-circle distance on the earth to radians.

\end{fulllineitems}

\index{waypoint() (pyesg.Arc method)}

\begin{fulllineitems}
\phantomsection\label{code:pyesg.Arc.waypoint}\pysiglinewithargsret{\bfcode{waypoint}}{\emph{k}}{}
Calculate the location of a selected point (lat, lon) according to:
+ the location of point 1 (lat1, lon1);
+ the location of point 2 (lat2, lon2);
+ the coefficient k decides the position between point 1 and point 2,

e.g.:
+ when k = 0.0, (lat, lon) is point 1;
+ when k = 0.5, (lat, lon) is the mid-point;
+ when k = 1.0, (lat, lon) is point 2.
{[}reference: \href{http://en.wikipedia.org/wiki/Great-circle\_navigation}{http://en.wikipedia.org/wiki/Great-circle\_navigation}{]}

\end{fulllineitems}


\end{fulllineitems}

\index{Interp (class in pyesg)}

\begin{fulllineitems}
\phantomsection\label{code:pyesg.Interp}\pysigline{\strong{class }\code{pyesg.}\bfcode{Interp}}
Interpolation algorithms.
\index{barycentric() (pyesg.Interp method)}

\begin{fulllineitems}
\phantomsection\label{code:pyesg.Interp.barycentric}\pysiglinewithargsret{\bfcode{barycentric}}{\emph{point}, \emph{triangle}}{}
(point, triangle) -\textgreater{} weight1, weight2, weight3

Barycentric Interpolation: interpolation in a triangle.

{[}reference: \href{https://classes.soe.ucsc.edu/cmps160/Fall10/}{https://classes.soe.ucsc.edu/cmps160/Fall10/}
resources/barycentricInterpolation.pdf{]}

\end{fulllineitems}

\index{regrid() (pyesg.Interp method)}

\begin{fulllineitems}
\phantomsection\label{code:pyesg.Interp.regrid}\pysiglinewithargsret{\bfcode{regrid}}{\emph{mesh\_old}, \emph{mesh\_new}, \emph{method='standard'}}{}
(mesh\_old, mesh\_new) -\textgreater{} matrix of weights and points indices.

Calculate the remapping coefficients (weights)
from an old mesh to a new mesh.
+ When method == standard,
the search algorithm can resolve any situation but slow;
+ When method == quick,
the situation is that mesh\_old and mesh\_new are very similar
to each other with some points nudged.

\end{fulllineitems}


\end{fulllineitems}

\index{Mesh (class in pyesg)}

\begin{fulllineitems}
\phantomsection\label{code:pyesg.Mesh}\pysiglinewithargsret{\strong{class }\code{pyesg.}\bfcode{Mesh}}{\emph{lat2d}, \emph{lon2d}}{}
Unstructed mesh grids, which is defined by
2 dimenional arrays lat2d and lon2d.

\end{fulllineitems}

\index{Point (class in pyesg)}

\begin{fulllineitems}
\phantomsection\label{code:pyesg.Point}\pysiglinewithargsret{\strong{class }\code{pyesg.}\bfcode{Point}}{\emph{lat}, \emph{lon}}{}
A point on the earth defined by the latitude and longitude (unit: degree).
\index{lat\_deg() (pyesg.Point method)}

\begin{fulllineitems}
\phantomsection\label{code:pyesg.Point.lat_deg}\pysiglinewithargsret{\bfcode{lat\_deg}}{}{}
radians -\textgreater{} degree

\end{fulllineitems}

\index{lon\_deg() (pyesg.Point method)}

\begin{fulllineitems}
\phantomsection\label{code:pyesg.Point.lon_deg}\pysiglinewithargsret{\bfcode{lon\_deg}}{}{}
radians -\textgreater{} degree

\end{fulllineitems}

\index{spherical\_coord() (pyesg.Point method)}

\begin{fulllineitems}
\phantomsection\label{code:pyesg.Point.spherical_coord}\pysiglinewithargsret{\bfcode{spherical\_coord}}{}{}
p (lat\_rad, lon\_rad) -\textgreater{} x, y, z

Return the (x, y, z) in a UNIT spherical coordinate system.

{[}reference: \href{http://en.wikipedia.org/wiki/Spherical\_coordinate\_system}{http://en.wikipedia.org/wiki/Spherical\_coordinate\_system}{]}

\end{fulllineitems}

\index{vector() (pyesg.Point method)}

\begin{fulllineitems}
\phantomsection\label{code:pyesg.Point.vector}\pysiglinewithargsret{\bfcode{vector}}{}{}
Return the vector from the center of the Earth to the point.

\end{fulllineitems}


\end{fulllineitems}

\index{Quadrangle (class in pyesg)}

\begin{fulllineitems}
\phantomsection\label{code:pyesg.Quadrangle}\pysiglinewithargsret{\strong{class }\code{pyesg.}\bfcode{Quadrangle}}{\emph{p1}, \emph{p2}, \emph{p3}, \emph{p4}}{}
An quadrangle on the earth defined by three points p1, p2, p3, and p4.
It also can be seen as the relationship between four points.

Note: p1 -\textgreater{} p2 -\textgreater{} p3 -\textgreater{} p4 should be rotative.
\index{angles() (pyesg.Quadrangle method)}

\begin{fulllineitems}
\phantomsection\label{code:pyesg.Quadrangle.angles}\pysiglinewithargsret{\bfcode{angles}}{}{}
Treated as two triangles.

\end{fulllineitems}

\index{area() (pyesg.Quadrangle method)}

\begin{fulllineitems}
\phantomsection\label{code:pyesg.Quadrangle.area}\pysiglinewithargsret{\bfcode{area}}{}{}
{[}reference: \href{http://mathworld.wolfram.com/SphericalPolygon.html}{http://mathworld.wolfram.com/SphericalPolygon.html}{]}

\end{fulllineitems}


\end{fulllineitems}

\index{Triangle (class in pyesg)}

\begin{fulllineitems}
\phantomsection\label{code:pyesg.Triangle}\pysiglinewithargsret{\strong{class }\code{pyesg.}\bfcode{Triangle}}{\emph{p1}, \emph{p2}, \emph{p3}}{}
An triangle on the earth defined by three points p1, p2, and p3.
It also can be seen as the relationship between three points.
\index{angles() (pyesg.Triangle method)}

\begin{fulllineitems}
\phantomsection\label{code:pyesg.Triangle.angles}\pysiglinewithargsret{\bfcode{angles}}{}{}
Calculate the included angle between two sides on the earth.
If we set a is the side p2-p3, b the side p3-p1, and c the side p1-p2.
Then the return value A is the included angle betwen sides b and c.

\end{fulllineitems}

\index{area() (pyesg.Triangle method)}

\begin{fulllineitems}
\phantomsection\label{code:pyesg.Triangle.area}\pysiglinewithargsret{\bfcode{area}}{}{}
Calculate the area of the triangle bounded by the sides made by the
three points p1 (lat1, lon1), p2 (lat2, lon2), and p3 (lat3, lon3)
according to the Girard's Theorem:
\$area = R\textasciicircum{}2 * E\$,
where R is the radius of the sphere, and E the angle excess:
\$E = A + B + C - pi\$.
Cosine rules are used to calculate the angles A, B, and C.

{[}references:
\href{http://www.princeton.edu/~rvdb/WebGL/GirardThmProof.html}{http://www.princeton.edu/\textasciitilde{}rvdb/WebGL/GirardThmProof.html}
\href{http://en.wikipedia.org/wiki/Spherical\_trigonometry}{http://en.wikipedia.org/wiki/Spherical\_trigonometry}
\href{http://mathforum.org/library/drmath/view/65316.html}{http://mathforum.org/library/drmath/view/65316.html}
{]}

\end{fulllineitems}


\end{fulllineitems}



\chapter{About}
\label{about::doc}\label{about:about}

\section{Author}
\label{about:author}

\section{License}
\label{about:license}
Unless stated otherwise, all files in the PyESG project, PyESG’s webpage (and wiki),
all images and all documentation including this User’s Guide are licensed using the
new BSD license:

\begin{Verbatim}[commandchars=\\\{\}]
Copyright (c) 2015, Feng Zhu
All rights reserved.

Redistribution and use in source and binary forms, with or without
modification, are permitted provided that the following conditions are met:

* Redistributions of source code must retain the above copyright notice, this
  list of conditions and the following disclaimer.

* Redistributions in binary form must reproduce the above copyright notice,
  this list of conditions and the following disclaimer in the documentation
  and/or other materials provided with the distribution.

* Neither the name of [project] nor the names of its
  contributors may be used to endorse or promote products derived from
  this software without specific prior written permission.

THIS SOFTWARE IS PROVIDED BY THE COPYRIGHT HOLDERS AND CONTRIBUTORS \PYGZdq{}AS IS\PYGZdq{}
AND ANY EXPRESS OR IMPLIED WARRANTIES, INCLUDING, BUT NOT LIMITED TO, THE
IMPLIED WARRANTIES OF MERCHANTABILITY AND FITNESS FOR A PARTICULAR PURPOSE ARE
DISCLAIMED. IN NO EVENT SHALL THE COPYRIGHT HOLDER OR CONTRIBUTORS BE LIABLE
FOR ANY DIRECT, INDIRECT, INCIDENTAL, SPECIAL, EXEMPLARY, OR CONSEQUENTIAL
DAMAGES (INCLUDING, BUT NOT LIMITED TO, PROCUREMENT OF SUBSTITUTE GOODS OR
SERVICES; LOSS OF USE, DATA, OR PROFITS; OR BUSINESS INTERRUPTION) HOWEVER
CAUSED AND ON ANY THEORY OF LIABILITY, WHETHER IN CONTRACT, STRICT LIABILITY,
OR TORT (INCLUDING NEGLIGENCE OR OTHERWISE) ARISING IN ANY WAY OUT OF THE USE
OF THIS SOFTWARE, EVEN IF ADVISED OF THE POSSIBILITY OF SUCH DAMAGE.
\end{Verbatim}


\chapter{Indices and tables}
\label{index:indices-and-tables}\begin{itemize}
\item {} 
\DUspan{xref,std,std-ref}{genindex}

\item {} 
\DUspan{xref,std,std-ref}{modindex}

\item {} 
\DUspan{xref,std,std-ref}{search}

\end{itemize}


\renewcommand{\indexname}{Python Module Index}
\begin{theindex}
\def\bigletter#1{{\Large\sffamily#1}\nopagebreak\vspace{1mm}}
\bigletter{p}
\item {\texttt{pyesg}}, \pageref{code:module-pyesg}
\end{theindex}

\renewcommand{\indexname}{Index}
\printindex
\end{document}
